\begin{center}
\thispagestyle{empty}
\vspace{-24em}
\includegraphics[trim=0 0 0 6cm,width=4cm]{images/logo.png}\\
\ \\
\vspace{-4em}
\LARGE{\textsc{Uniwersytet Jagielloński}}\\[-0.9ex]
\ \\
\Large{\textsc{Informatyka Analityczna}}\\[2ex]
\ \\
\ \\
\Huge{\textbf{Problem Range Mode Query}}
\ \\
\ \\
\ \\
\ \\
\Large{Leopold Kozioł}
\ \\
\ \\
\ \\
\ \\
\ \\
\ \\
\ \\

\Large{Opiekun pracy}\\
\Large{dr Lech Duraj}\\
\ \\
\ \\
\ \\
\ \\
\ \\
\ \\
\Large{wrzesień 2021}
\end{center}


\newpage
\thispagestyle{plain}
\begin{abstract}
\hspace{0.12\textwidth}\begin{minipage}[t]{0.6\textwidth}
    W problemie \textsc{Range Mode Query}, dana jest na wejściu $n$-elementowa tablica liczb naturalnych $A$ oraz $q$ przedziałów tablicy $A$. Dla każdego danego na wejściu przedziału chcemy policzyć jego dominantę, czyli element który najczęściej występuje na danym fragmencie.\\
    Praca ta ma celu implementację i porównanie różnych algorytmów i struktur danych rozwiązujących problem \textsc{Range Mode Query}.
\end{minipage}
\end{abstract}



\newpage
\tableofcontents